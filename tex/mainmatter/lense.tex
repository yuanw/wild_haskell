%%%
\chapter{Learning Lens}


The content of this chapter comes fron Simon Peyton Jones's \href{https://skillsmatter.com/skillscasts/4251-lenses-compositional-data-access-and-manipulation}{Lenses: Compositional data access and mainpulation} talk.

\section{Native Approach}

\begin{minted}{haskell}
data Person = Person { name    :: String,
                     , address :: Address
                     , salary  :: Int }

data Address = Address { road :: String
                       , city :: String
                       , zip  :: String }
\end{minted}

The goals for lens

* To access and update a given field
* Compose Lense

\begin{minted}{haskell}
data LensR s a = LensR { viewR :: s -> a
                       , setR  :: a -> s -> s }
\end{minted}

\begin{minted}{haskell}
data Prisms s a = Prisms { match :: s -> Either s a
                         , build :: a -> s}
\end{minted}

Type `s` represent the overall record we try to access or update, Type `a` represent the field we are try to acccess or update.

How to compose `LensR`

\begin{minted}{haskell}
composeL :: LensR s1 s2 -> LensR s2 a -> LensR s1 a
composeL (L v1 s1) (L v2 s2)
    = L (\ s -> v2 (v1 s))
        (\ a s -> s1 (s2 a (v1 s)) s)
\end{minted}

The two big problems with this approach is to inefficent, and inflexiable.

\subsection{A step further on this Native Approach}

\begin{minted}{haskell}
data LensR s a
    = L { viewR :: s -> a
        , setR  :: a -> s -> a
        , mod   :: (a -> a) -> s -> s
        , modM  :: (a -> Maybe a) -> s -> Maybe s
        , modIO :: (a -> IO a) -> s -> IO s }
\end{minted}

The last two (or three if you know your functor well) share lots of commonality.

If we can abstract the common pattern

\begin{minted}{haskell}
data LensR s a
    = L { viewR :: s -> a
        , setR  :: a -> s -> a
        , mod   :: (a -> a) -> s -> s
        , modF  :: Functor f => (a -> f a) -> s -> f s }
\end{minted}



\begin{minted}{haskell}
type Lens' s a = forall f. Functor f => (a -> f a) -> s -> f s
\end{minted}

You may think `Lens'` looks nothing like `LensR`, but they are actually isomorphic.



Side Bar :: Isomorphic

If `A` and `B` are isomorphic, it means we are getting `B` to `A` without lost any information, and vice verse.

Before proving `Lens'` and `LensR` is isomorphic, allow me introduce two oddly looking functors `Const` and `Identity`.


Break Down::
\begin{minted}{haskell}
view :: Lens' s a -> s -> a => ((a -> f a) -> s -> f s) -> s -> a
view ln s = getConst (ln const s)
\end{minted}

\begin{minted}{haskell}
let fred = P {_name = "Fred", _salary = 100}
name :: Lens' Person String
name fn (P n s) -> (\ n' -> P n' s) <$> fn n
\end{minted}

\begin{minted}{haskell}
view name fred
= (getConst . name . const) fred
= getConst ((\n' -> P n' s) <$> Const "Fred")
= getConst . Const "Fred"
= "Fred"
\end{minted}

\begin{minted}{haskell}
set name "John" fred
= getID $ name (Id . Const  "John") fred
= getID $ (\ n' -> P n' 100) <$> ((ID. Const  "John" ) fred)
= getID $ (\ n' -> P n' 100) <$> Id "John"
= getID $ ID (P "John" 100 )
= P "John" 100
\end{minted}


Compose Lens'

Function application

\url{https://www.youtube.com/watch?v=sfWzUMViP0M&t=441s}

\url{https://www.google.com/url?sa=t&rct=j&q=&esrc=s&source=web&cd=1&ved=2ahUKEwiS6dD65PfdAhVSJjQIHRzzC9IQFjAAegQIBBAC&url=http%3A%2F%2Fwww.cs.ox.ac.uk%2Fpeople%2Fjeremy.gibbons%2Fpublications%2Fpoptics.pdf&usg=AOvVaw3KgxUg3x37WIToXgmRu79C}

The problem: heterogeneous composite data

Worse: consider access to the A in a composiation data structure
Maybe (A x B) built using both sums and products.

Sum part comes with Maybe `data Maybe a = Just a | Nothing`


\begin{minted}{haskell}
*Main Lib> import Control.Lens
*Main Lib Control.Lens> ("hello","world") ^. _2
"world"
*Main Lib Control.Lens>
\end{minted}

\section{What is a Lens ?}
Lenses address some part of a “structure” that always exists, either look that part, or set that part.
“structure” can be a computation result, for example, the hour in time. Functional setter and getter.
  Data.Lens


\begin{minted}{haskell}
data Lens s a = Lens { set  :: s -> a -> a
                     , view :: s -> a
                     }
view :: Lens s a -> s -> a
set  :: Lens s a -> s -> a -> s
\end{minted}

view looks up an attribute a from s
view is the getter, and set is the setter.
Laws
1. set l (view l s) s = s
2. view l (set l s a) = a
3. set l (set l s a) b = set l s b
Law 1 indicates lens has no other effects
since set and view in Lens both start with s ->, so we can fuse the two functions into a single function
`s -> (a -> s, a)`.

So we could define Lens as
\begin{minted}{haskell}
data Lens s a = Lens (s -> (a -> s), a)
data Store s a = Store (s -> a) s
data Lens s a = Lens (s -> Store a s)
\end{minted}

Side bar Store Comonad
url{https://stackoverflow.com/questions/8428554/what-is-the-comonad-typeclass-in-haskell}
hurl{ttps://stackoverflow.com/questions/8766246/what-is-the-store-comonad}

Lens can form a category

 Semantic Editor Combinator

The Power is in the dot
\begin{minted}{haskell}
(.)         :: (b -> c) -> (a -> c) -> (a -> c)
(.).(.)     :: (b -> c) -> (a1 -> a2 -> b) -> a1 -> a2 -> c
(.).(.).(.) :: (b -> c) -> (a1 -> a2 -> a3 -> b) -> a1 -> a2 -> a3 -> c
\end{minted}

\href{https://www.reddit.com/r/haskellquestions/comments/ayi445/help_me_understand_the_function_and_its_type}{Detail Explantation}

\begin{minted}{haskell}
(.) :: (b -> c) -> (a -> b) -> a -> c
f . g = \ t -> f (g t)
\end{minted}


\begin{align*}
& ((.).(.)) (+1) (+) 10 10 = 21 \\
&  ((.).(.)) (+1) (+) 10 10 \\
& =  (.)((.) (+1)) (+) 10 10 \\
& =  ((.) (+1)) (+ 10) 10   \\
& = ((+1). (+10)) 10 \\
& = (+1) ((+10) 10) \\
& = (+1) (10 + 10) \\
& = 21
\end{align*}

\begin{minted}{haskell}
(.).(.) :: (b -> c) -> (a -> b) -> a -> c
dotF . dotG :: (b -> c) -> (a -> c) -> a -> c
dotF :: b -> c
dotG :: a -> b
\end{minted}

since \textbf{dotF} is just an alias to  \textbf{(.)}
\begin{equation}
dotF :: (u -> v) -> (s -> u) -> s -> v
(u -> v) -> (s -> u) -> s -> v === b -> c
\end{equation}
therefore

\begin{equation}
b === (u -> v) \\
c === (s -> u) -> s -> v
\end{equation}

`dotG` is also an alias to `(.)`

\begin{equation}
dotG :: (y -> z) -> (x -> y) -> x -> z
(y -> z) -> (x -> y) -> x -> z === a -> b
\end{equation}
therefore

\begin{equation}
a === (y -> z)
b === (x -> y) -> x -> z
\end{equation}

since \textbf{b} appears on both side

\begin{equation}
a === y -> z
b === (u -> v)
b === (x -> y) -> x -> z
c === (s -> u) -> s -> v
\end{equation}

we can deduct

\begin{align*}
u &=== x -> y  \\
v &=== x -> z
\end{align*}

therefore \begin{equation}c === (s -> x -> y) -> s -> x -> z\end{equation}

\begin{align*}
(.).(.) & === a -> c \\
(.).(.) & === (y -> z) -> (s -> x -> y) -> s -> x - z
\end{align*}


we can generalize this to any functor

\begin{minted}{haskell}
fmap                :: Functor f  => (a -> b) -> f a -> f b
fmap . fmap         :: ( Functor f, Functor g)   => (a -> b) -> f (g a) -> f (g b)
fmap . fmap . fmap  :: (Functor f, Functor g, Functor h) => (a -> b) -> f (g (h a)) -> f (g (h b))
\end{minted}

it means we compose a function under arbitrary level deep nested context
Semantic Editor Combinator

`type SEC s t a b = (a -> b) -> s -> t`
it like a functor (a -> b) -> f a -> f b
`fmap` is Semantic Editor Combinator
`fmap . fmap` is also a SEC

so we use `s` to generalize `f a`,  `f (g a)`, `f (g (h a))` and `t` to generalize `f b`,  `f (g b)`, `f (g (h b))`.
It may seems counterintuitive at first, we lost the relation between a and s, b and t.
Functor is a semantic Editor Combinator
`fmap :: Functor f => SEC (f a) (f b) a b`

first is a also SEC ?
\begin{minted}{haskell}
first :: SEC (a, c) (b,c) a b
first f (a, b) = (f a, b)
\end{minted}
Setters
We can compose Traversable the way as we can compose `(.)` and `fmap`

\begin{minted}{haskell}
traverse                       :: (Traversable f, Applicative m) => (a -> m b) -> f a -> m (f b)
traverse . traverse            :: (Traversable f, Traversable g, Applicative m) => (a -> m b) -> f (g a) -> m (f (g b))
traverse . traverse . traverse :: (Traversable f, Traversable g, Traversable h, Applicative m) => (a -> m b) -> f (g (h a)) -> m (f (g (h b)))
\end{minted}

traverse is a generalized version of mapM, and work with any kind of Foldable not just List.

\begin{minted}{haskell}
class (Functor f, Foldable f) => Traversable f where
   traverse :: Applicative m => (a -> m b) -> f a -> m (f b)
\end{minted}

\begin{minted}{haskell}
fmapDefault :: forall t a b. Traverable t => (a -> b) -> t a -> t b
fmapDefault f = runIndentity . traverse (Identity . f)
\end{minted}

build `fmap` out from traverse
we can change `fmapDefault`

\begin{minted}{haskell}
over l f = runIdentity . l (Identity . f)
over traverse f = runIdentity . traverse (Identity . f)
                = fmapDefault f
                = fmap f
\end{minted}

type of `over :: ((a -> Identity b) -> s -> Identity t) -> (a -> b) -> s -> t`
type Setter s t a b = (a -> Identity b) -> s -> Identity t
so we could rewrite `over` as
over :: Setter s t a b -> (a -> b) -> s -> t
let’s apply setter

\begin{minted}{haskell}
mapped :: Functor f => Setter (f a) (f b) a b
mapped f = Identity . fmap (runIdentity . f)
over mapped f = runIdentity . mapped (Identity . f)
              = runIdentity . Identity . fmap (runIdentity . Identity . f)
              - fmap f
\end{minted}

Examples

\begin{minted}{haskell}
over mapped (+1) [1,2,3]  ===> [2,3,4]
over (mapped . mapped) (+1) [[1,2], [3]] ===> [[2,3], [4]]
chars :: (Char -> Identity Char) -> Text -> Identity Text
chars f = fmap pack . mapped f . unpack
\end{minted}

Laws for setters
Functor Laws:
1. `fmap` id = id
2. fmap f . fmap g = fmap (f . g)

Setter Laws for a legal Setter l.
1. over l id = id
2. over l f . over l g = over l (f . g)


Practices
Simplest lens
\verb|(1,2,3) ^. _2   ===> 2|
\verb|view _2 (1, 2, 3)|


\todo{Third Program: Upload Spreadsheet to google bigquery}

In this chapter, we are going to write a command line program to upload a spreadsheet file's content to Google BigQuery. There is awesome library Gogol provides Haskell binding to Google API.



We need gogol 0.3.0 or higher, Make sure your project's resolver version is lts-9.0 or later. You can verify it by doing `stack list-dependencies`.

Starting Point

So how to use Gogol library ? The lib provides an [example](https://github.com/brendanhay/gogol/blob/develop/examples/src/Example/Storage.hs) for Google Cloud Storage.

Let's do a little change, so it fetch all the bigquery project the current default google credential has access to.

You need to install gcloud, and setup default
\begin{minted}{bash}
gcloud init
gcloud auth application-default login
\end{minted}

and let's put the following code into our `Main.hs`.

\begin{minted}{haskell}
module Main where

import Control.Lens                 ((&), (.~), (<&>), (?~))
import Control.Monad.Trans.Resource (runResourceT)
import Control.Monad.IO.Class

import System.IO (stdout)
import Network.Google.Auth   (Auth, Credentials (..), initStore)
import qualified Network.Google         as Google
import qualified Network.Google.BigQuery as BigQuery
import Network.HTTP.Conduit (Manager, newManager, tlsManagerSettings)
import           Network.Google.Auth.Scope       (AllowScopes (..),                                                  concatScopes)

example :: IO BigQuery.ProjectList
example = do
    lgr  <- Google.newLogger Google.Debug stdout
    m <- liftIO (newManager tlsManagerSettings) :: IO Manager
    c <- Google.getApplicationDefault m
 -- Create a new environment which will discover the appropriate
 -- AuthN/AuthZ credentials, and explicitly state the OAuth scopes
 -- we will be using below, which will be enforced by the compiler:
     env  <- Google.newEnvWith c lgr m <&>
           (Google.envLogger .~ lgr)
         . (Google.envScopes .~ BigQuery.bigQueryScope)
    runResourceT . Google.runGoogle env $ Google.send BigQuery.projectsList

 main :: IO ()
 main = do
     projects <- example
     print projects
\end{minted}

You need add following build depends to make stack build successed.

\begin{minted}{yaml}
  build-depends:       base >= 4.7 && < 5
                     , bytestring
                     , conduit
                     , conduit-extra
                     , gogol
                     , gogol-bigquery
                     , gogol-core
                     , http-conduit
                     , lens
                     , resourcet
\end{minted}

Quite few things need to unpack here.

Create a bigquery dataset


\begin{minted}{haskell}
{-# LANGUAGE RankNTypes      #-}
{-# LANGUAGE TemplateHaskell #-}

module Main where

import           Control.Lens

data Point = Point
  { _postionX :: Double
  , _postionY :: Double} deriving (Show)

makeLenses ''Point

data Segment = Segment {
  _segmentStart :: Point,
  _segmentEnd   :: Point
} deriving (Show)
makeLenses ''Segment

makePoint :: (Double , Double) -> Point
makePoint = uncurry Point

makeSegment :: (Double, Double) -> (Double, Double) -> Segment
makeSegment start end = Segment (makePoint start) (makePoint end)

updateSegment :: Segment -> Segment
updateSegment = (segmentStart .~ makePoint (10, 10)) . (segmentEnd .~ makePoint (10, 10))
\end{minted}



\url{http://www.scs.stanford.edu/16wi-cs240h/slides/concurrency-slides.html#(1)}

\href{https://hackage.haskell.org/package/base-4.10.0.0/docs/Control-Exception.html#v:catchJust}{catchJust]}

\todo{Add exception handle in the guess number program}




 References
\href{https://skillsmatter.com/skillscasts/4251-lenses-compositional-data-access-and-manipulation}{SPJ Lenses: compositional data access and manipulation}
\href{https://www.youtube.com/watch?v=cefnmjtAolY}{Edward Kmett's NYC Haskell Meetup talk}
\href(http://comonad.com/haskell/Lenses-Folds-and-Traversals-NYC.pdf){Edward Kmett's NYC Haskell Meetup talk slide}
\url{http://hackage.haskell.org/package/lens-tutorial-1.0.3/docs/Control-Lens-Tutorial.html}
\url{https://www.youtube.com/watch?v=H01dw-BMmlE}
\url{https://www.youtube.com/watch?v=QZy4Yml3LTY}
\url{https://www.youtube.com/watch?v=T88TDS7L5DY}
\url{http://lens.github.io/tutorial.html}
\url{https://www.reddit.com/r/haskell/comments/9ded97/is_learning_how_to_use_the_lens_library_worth_it/e5hf9ai/}
\url{https://blog.jle.im/entry/lenses-products-prisms-sums.html}
