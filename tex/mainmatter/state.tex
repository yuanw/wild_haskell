%%%
\chapter{From State to Monad Transformers}

\section{State Monad}

First, let's look what is State.

\mint{haskell}|newtype State s a = State { runState :: s-> (a, s)}|


In case you are not familiar with \href{https://wiki.haskell.org/Newtype}{newtype}.
It is just like `data` but it can only *has exactly one constructor with exactly one field in it.* In this case `State` is the constructor and `runState` is the single field. The type of `State` constructor is `(s ->  a, s)) -> State s a`, since `State` uses record syntax, we have a filed accessor `runState` and its type is State s a -> s -> (a, s).
"A State is a function from a state value" to (a produced value <strong><em>a</em></strong>, and a resulting state <strong><em>s</em></strong>).</blockquote>
An importance takeaway which might not so oblivious to Haskell beginner is that: `State` is merely a function with type `s -> (a,s)`, (a function wrapped inside constructor `State`, to be exact, and we can unwrap it using `runState`).

Secondly, `State` function takes a `s`representing a state, and produces a tuple contains value `a` and new state `s`.

 exec

The first exercise is to implement `exec` function, it takes a `State` function and initial `state` value, returns the new state.

It is pretty straightforward, we just need to apply `State` function with `state` value, and takes the second  element from the tuple.

A simple solution can be

\begin{minted}{haskell}
exec :: State s a -> s -> s
exec f initial = (\ (_, y) -> y) (runState f initial)
\end{minted}

Since the state value `initial` appears on both sides of `=`, we rewrite it a little more point-free. `runState f` is a partial applied function takes `s` returns `(a,s)`

\begin{minted}{haskell}
exec f = (\(_, y) -> y) . runState f
\end{minted}

%In `Prelude`, there is a [`snd`](http://hackage.haskell.org/package/base-4.11.1.0/docs/Prelude.htmlv:snd) function does exactly the same thing of `\(_, y) -> y)`.

\begin{minted}{haskell}
exec f = P.snd . runState f
\end{minted}

We could also write `exec`

`exec (State f) = P.snd . f`


\subsection{Basic Usage of State}

\begin{minted}{haskell}
module Dice where

import Control.Applicative
import Control.Monad.Trans.State
import System.Random

rollDiceIO :: IO (Int, Int)
rollDiceIO = liftA2 (,) (randomRIO (1, 6)) (randomRIO (1,6))

rollNDiceIO :: Int -> IO [Int]
rollNDiceIO 0 = pure []
rollNDiceIO count = liftA2 (:) (randomRIO (1, 6)) (rollNDiceIO (count - 1))

clumsyRollDice :: (Int, Int)
clumsyRollDice = (n, m)
    where
        (n, g) = randomR (1, 6) (mkStdGen 0)
        (m, _) = randomR (1, 6) g

-- rollDice :: StdGen -> ((Int, Int), StdGen)
-- rollDice g = ((n, m), g'')
--     where
--         (n, g') = randomR (1, 6) g
--         (m, g'') = randomR (1, 6) g'

-- use state to construct
rollDie :: State StdGen Int
rollDie = state $ randomR (1, 6)

-- use State as Monad
rollDieM :: State StdGen Int
rollDieM = do generator <- get
              let (value, generator') = randomR (1, 6) generator
              put generator'
              return value

rollDice :: State StdGen (Int, Int)
rollDice = liftA2 (,) rollDieM rollDieM

rollNDice :: Int -> State StdGen [Int]
rollNDice 0 = state (\s -> ([], s))
rollNDice count = liftA2 (:) rollDieM (rollNDice (count - 1))
\end{minted}

How about draw card from a deck

\begin{minted}{haskell}
module Deck where

import Control.Applicative
import Control.Monad.Trans.State
import Data.List
import System.Random

data Rank = One | Two | Three | Four | Five | Six | Seven | Eight | Nine | Ten | Jack | Queue | King deriving (Bounded, Enum, Show, Eq, Ord)

data Suit = Diamonds | Clubs | Hearts | Spades deriving (Bounded, Enum, Show, Eq, Ord)

data Card = Card Suit Rank deriving (Show, Eq, Ord)

type Deck = [Card]

fullDeck :: Deck
fullDeck = [Card suit rank  | suit <- enumFrom minBound,
                              rank <- enumFrom minBound]

removeCard :: Deck -> Int -> Deck
removeCard [] _ = []
removeCard deck index = deck' ++ deck''
    where (deck', remain) = splitAt (index + 1) deck
          deck''    = drop 1 remain


drawCard :: State (StdGen, Deck) Card
drawCard = do (generator, deck) <- get
              let (index, generator') = randomR (0, length deck ) generator
              put (generator', removeCard deck index)
              return $ deck !! index

drawNCard :: Int -> State (StdGen, Deck) [Card]
drawNCard 0 = state (\s -> ([], s))
drawNCard count = liftA2 (:) drawCard (drawNCard $ count - 1)
\end{minted}

How about folding a list using `State`
https://github.com/yuanw/applied-haskell/blob/2018/monad-transformers.mdhow-about-state


\begin{minted}{haskell}
foldState :: (b -> a -> b) -> b -> [a] -> b
foldState f accum0 list0 =
    execState (mapM_ go list0) accum0
  where
    go x = modify' (\accum -> f accum x)
\end{minted}

\section{Monad Transformers}
\subsection{Motivation} Why we cannot compose any two monad

\begin{minted}{haskell}
import Control.Applicative

newtype Compose f g a = Compose (f (g a))

instance (Functor f, Functor g) => Functor (Compose f g) where
    fmap f (Compose h) = Compose ((fmap . fmap) f h)


instance (Applicative f, Applicative g) => Applicative (Compose f g) where
    pure = Compose . pure . pure
    (Compose h)  <*> (Compose j) = Compose $ pure (<*>) <*> h <*> j

instance (Monad f, Monad g) => Monad (Compose f g) where
     (Compose fga) >>= m = Compose $ fga >>= \ ga -> let gfgb = ga >>= (return . m) in undefined
\end{minted}

https://stackoverflow.com/questions/7040844/applicatives-compose-monads-dont


\subsection{Build a composed monad by hand}


\begin{minted}{haskell}
newtype StateEither s e a = StateEither
    { runStateEither :: s -> (s, Either e a)
    }
\end{minted}

Let's implement the functor instance of this typeP


 References
 \url{https://en.wikibooks.org/wiki/Haskell/Understanding_monads/State}
 \url{https://haskell.fpcomplete.com/library/rio}
